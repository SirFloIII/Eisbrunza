\documentclass[a4paper,11pt,notitlepage,fullpage]{article}
%\documentclass{report}

\usepackage{fullpage}
\usepackage[utf8]{inputenc}
%\usepackage[ngerman]{babel}
%\usepackage[english]{babel}
\usepackage{amsmath}
\usepackage{amssymb}
\usepackage{latexsym}
\usepackage{mathtools}
\usepackage{listings}
\usepackage{bbm}
%\usepackage{algorithm}
%\usepackage{algpseudocode}
\usepackage{graphicx}
\usepackage{booktabs}
\usepackage{hhline}
\usepackage{amsthm}
\usepackage{cite}
\usepackage{wrapfig}
\usepackage{hyperref}
\usepackage{titling}
\usepackage{color}

\setlength{\droptitle}{-60pt}

\newcommand{\R}{\mathbb R}
\newcommand{\p}{\mathbb P}
\newcommand{\pp}[1]{\mathbb P\left[#1\right]}
\newcommand{\E}[1]{\mathbb E\left[#1\right]}
\newcommand{\V}{\mathbb V}
\newcommand{\Vv}[1]{\mathbb V\left[#1\right]}
\newcommand{\Cov}[1]{\mathbb Cov\left[#1\right]}
\newcommand{\F}{\mathcal{F}}
\newcommand{\ind}{\mathbbm{1}}
\newcommand{\indd}[1]{\mathbbm{1}_{#1}}
\newcommand{\norm}[2]{\left|\left|{#1}\right|\right|_{#2}}
\DeclareMathOperator*{\limm}{l.\hspace{-0.18em}i.\hspace{-0.19em}m}
\DeclareMathOperator*{\spann}{span}

\begin{document}
\author{Florian Bogner \& Alexander Palmrich}
\title{Stochastische Prozesse - Übung 11}
\maketitle

\begin{enumerate}
\setcounter{enumi}{46}

%%für ein Bild das copy-pasten und reinkommentieren
%\begin{figure}[h!]
%\centering
%\includegraphics[width=0.9\textwidth]{gfx/bildname.png}
%\label{fig1}
%\caption{TODO Beschreibung des Bildes}
%\end{figure}

%01
\item Laut Folie 129 ist das AR-Polynom $q(z) := 1 - a z^4 \neq 0$ für $|z| \leq 1$ da $|a| < 1$. Damit ist die Stabilitätsbedingung erfüllt. Wir tanzen die Vorlesungsfolie 110 nach: 
\begin{align*}
x_t &= a x_{t-4} + \epsilon_t \\
&= a^2 x_{t-8} + a \epsilon_{t-4} + \epsilon_t \\
&= a^3 x_{t-12}+ a^2 \epsilon_{t-8} + a \epsilon_{t-4} + \epsilon_t \\
&= \vdots \\
&= \limm_{k \to \infty}\left( a^k x_{t-4k} + \sum_{j=0}^{k-1}a^j \epsilon_{t-4j} \right) \\
&= 0 + \sum_{j\geq0} a^j \epsilon_{t-4j}
\end{align*}
Damit haben wir unserern Lösungskandiaten gefunden. Probe:
\begin{align*}
x_t = \sum_{j\geq0} a^j \epsilon_{t-4j} = \epsilon_t + a\sum_{j=0}^{\infty} a^j \epsilon_{t-4-4j} = a x_{t-4} + \epsilon_t
\end{align*}
Hurra. ACF:
\begin{align*}
\gamma(k) &= \langle x_t, x_{t-k} \rangle \\
&= \sum_{j\geq0} \sum_{i\geq0} \langle a^j \epsilon_{t - 4j}, a^{i-k} \epsilon_{t - k - 4i}\rangle \\
&= \sum_{j\geq0} \sum_{i\geq0} a^{j+i-k} \langle \epsilon_{t - 4j}, \epsilon_{t - k - 4i}\rangle \\
&= \sum_{j\geq0} \sum_{i\geq0} a^{j+i-k} \delta_{4j}^{k+4i} \sigma^2 \\
&= \begin{cases}
0 & k \neq 0 \mod 4 \\
\sigma^2 \sum_{j\geq0} a^{2j-k} & k = 0 \mod 4 \\
\end{cases} \\
&= \begin{cases}
0 & k \neq 0 \mod 4 \\
\sigma^2 \frac{a^{-|k|}}{1 - a^2} & k = 0 \mod 4 \\
\end{cases} \\
\end{align*}



%02
\item Das AR-Polynom von $x_t = 0.1 x_{t-1} - 0.2 x_{t-2} + 0.3 x_{t-3} + \epsilon_t$ ist $$q(z) = 0.1 z - 0.2 z^2 + 0.3 z^3 + 1$$.
Wir wenden die umgekehrte Dreiecksungleichung an und schätzen $|z| \leq 1$ ab:
$$|q(z)| \geq -(0.1 + 0.2 + 0.3) + 1 = 0.4$$
Damit hat $q(z)$ keine Nullstellen in der Einheitskreisscheibe und die Stabilitätsbedingung ist erfüllt.
\begin{align*}
\end{align*}


%03
\item bla
\begin{enumerate}
%a
\item a
\begin{align*}
\end{align*}

%b
\item b
\begin{align*}
\end{align*}
\end{enumerate}

%04
\item bla
\begin{enumerate}
%a
\item a
\begin{align*}
\end{align*}

%b
\item b
\begin{align*}
\end{align*}
\end{enumerate}

%05
\item bla
\begin{enumerate}
%a
\item a
\begin{align*}
\end{align*}

%b
\item b
\begin{align*}
\end{align*}
\end{enumerate}

\end{enumerate}



\end{document}
