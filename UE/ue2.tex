\documentclass[a4paper,11pt,notitlepage,fullpage]{article}
%\documentclass{report}

\usepackage{fullpage}
\usepackage[utf8]{inputenc}
%\usepackage[ngerman]{babel}
\usepackage[english]{babel}
\usepackage{amsmath}
\usepackage{amssymb}
\usepackage{latexsym}
\usepackage{mathtools}
\usepackage{listings}
\usepackage{algorithm}
\usepackage{algpseudocode}
\usepackage{graphicx}
\usepackage{booktabs}
\usepackage{hhline}
\usepackage{amsthm}
\usepackage{cite}
\usepackage{wrapfig}
\usepackage{hyperref}
\usepackage{titling}
\usepackage{color}
\usepackage{bbm}
\newcommand{\E}{\mathbb{E}}
\newcommand{\p}{\mathbb{P}}
\newcommand{\F}{\mathcal{F}}
\newcommand{\V}{\mathbb Var}
\newcommand{\C}{\mathbb Cov}

\setlength{\droptitle}{-60pt}


\begin{document}
\author{Florian Bogner \& Alexander Palmrich}
\title{Stochastische Prozesse - Übung 2}
\maketitle

\begin{enumerate}
\setcounter{enumi}{5}
%6
\item Sei $Z\sim N(0,1)$ und $C(t)=\sqrt{t}Z$ für $t\geq0$.
\begin{enumerate}
%a
\item
\begin{align*}
\E[C(t)] &= \E[\sqrt{t}Z]\\
&=\sqrt{t}\E[Z]\\
&=\sqrt{t}\cdot0\\
&= 0\\
\\
\E[C(t)^2] &= \E[tZ^2]\\
&=t\E[Z^2]\\
&=t\cdot1\\
&=t
\end{align*}

%b
\item Mit dem Reproduktionssatz für $n = 1, m \in \mathbb N$:
\begin{align*}
\vec C := \begin{pmatrix}
C(t_1) \\ C(t_2) \\ \vdots \\ C(t_m)
\end{pmatrix} &=
\begin{pmatrix}
\sqrt{t_1} \\ \sqrt{t_2} \\ \vdots \\ \sqrt{t_m}
\end{pmatrix} \cdot (Z)
\end{align*}
ergo:
\begin{align*}
\vec C &\sim N\left(0, \begin{pmatrix}
\sqrt{t_1} \\ \sqrt{t_2} \\ \vdots \\ \sqrt{t_m}
\end{pmatrix} \cdot 1 \cdot \begin{pmatrix}
\sqrt{t_1} & \sqrt{t_2} & \hdots & \sqrt{t_m}
\end{pmatrix}\right) \\
&\sim N\left(0, \begin{pmatrix}
t_1 & \sqrt{t_1t_2} & \hdots & \sqrt{t_1t_m} \\
\sqrt{t_1t_2} & t_2 & \hdots & \sqrt{t_2t_m} \\
\vdots & \vdots & \ddots & \vdots \\
\sqrt{t_1t_m} & \sqrt{t_2t_m} & \hdots & t_m
\end{pmatrix}\right)
\end{align*}
Die endlichdimensionalen Randverteilungen sind normalverteilt, also handelt es sich um einen Gaußprozess. Man beachte, das die Kovarianzmatrix hoch-singulär ist, also nur Rang 0 hat.

%c
\item Keine Brownsche Bewegung, da der Prozess durch einen einzigen Zeitpunkt für die gesamte Zukunft vorherbestimmt ist. Damit können Inkremente nicht mehr unabhängig sein.\\
Oder: Pfade sind differenzierbar, das dürfte aber nur mit $\p=0$ passieren.
\end{enumerate}

%7
\item Sei $(W_t, t\geq0)$ standard Brownsche Bewegung, $(\F(t), t\geq0)$ erzeugte Filtration.
\begin{align*}
\E[W_1^2W_3 | \F_2] &= W_1^2\E[W_3 | \F_2]  \\
&= W_1^2\E[W_3 -W_2 +W_2 | \F_2]  \\
&= W_1^2(\E[W_3 -W_2 | \F_2] +\E[W_2 | \F_2])  \\
&= W_1^2(\E[W_3 -W_2] +W_2)  \\
&= W_1^2(0 +W_2)  \\
&= W_1^2W_2  \\
\\
\E[W_1^2W_3] &= \E[W_1^2(W_3-W_1+W_1)]  \\
&= \E[W_1^2(W_3-W_1)]+\E[W_1^2W_1]  \\
&= \E[W_1^2]\E[W_3-W_1]+\E[W_1^3]  \\
&= \E[W_1^2] \cdot 0 + 0  \\
&= 0  \\
%\\
%\E[W_1W_2W_3] &= \E[W_1(W_2-W_1+W_1)W_3]\\
%&= \E[W_1(W_2-W_1)W_3]+\E[W_1^2W_3]\\
%&= \E[W_1]\E[W_2-W_1]\E[W_3]+\E[W_1^2W_3]\\
\\
\E[W_1W_2W_3] &= \E[W_1W_2(W_3-W_2+W_2)]  \\
&= \E[W_1W_2(W_3-W_2)]+\E[W_1W_2^2]  \\
&= \E[W_1W_2]\E[(W_3-W_2)]+\E[W_1W_2^2]  \\
&= \E[W_1W_2]\cdot0+\E[W_1W_2^2]  \\
&= \E[W_1(W_2-W_1+W_1)^2]  \\
&= \E[W_1(W_2-W_1)^2 + 2W_1^2 (W_2-W_1) + W_1^3]  \\
&= \E[W_1(W_2-W_1)^2] + \E[2W_1^2 (W_2-W_1)] + \E[W_1^3]  \\
&= 0 \cdot \E[(W_2-W_1)^2] + \E[2W_1^2]\cdot 0 + 0 \\
&= 0 \\
\\
\E[W_1W_2W_3W_4] &= \E[W_1W_2W_3(W_4-W_3+W_3)]  \\
&= \E[W_1W_2W_3(W_4-W_3)] + \E[W_1W_2W_3^2]  \\
&= \E[W_1W_2W_3] \cdot 0 + \E[W_1W_2W_3^2]  \\
&= \E[W_1W_2(W_3 - W_2 + W_2)^2] \\
&= \E[W_1W_2((W_3-W_2)^2 + 2(W_2(W_3-W_2) + W_2^2)]  \\
&= \E[W_1W_2(W_3-W_2)^2] + 2\E[W_1W_2^2(W_3-W_2)] + \E[W_1W_2^3]  \\
&= \E[W_1W_2]\E[(W_3-W_2)^2] + 2\E[W_1W_2^2] \cdot 0 + \E[W_1(W_2-W_1+W_1)^3] \\
&= 1 \cdot 1 + 0 + \E[W_1((W_2 - W_1)^3 + 3(W_2 - W_1)^2W_1 + 3(W_2 - W_1)W_1^2 + W_1^3] \\
&= 1 + 0 \cdot 0 + 3\E[W_1^2 (W_2-W_1)^2] + 3 \cdot 0 \cdot \E[W_1^3] + \E[W_1^4] \\
&= 1 + 3(1 \cdot 1) + 0 + 3 \\
&= 7
\end{align*}

\newpage
%8
\item $X_t = W_t^4 - 3t^2$
\begin{enumerate}
\item Sei $t \geq 0$:
\begin{align*}
\E[X_t] &= \E[W_t^4] - 3t^2 = 3t^2 - 3t^2 = 0 \\
\V[X_t] &= \E[X_t^2] - \E[X_t]^2 \\
&= \E[W_t^8 - 6W_t^4 t^2 + 9t^4] - 0 \\
&= \E[W_t^8] - 6t^4\E[W_t^4] + 9t^4 \\
&= 105 t^4 - 18 t^4 + 9t^4 \\
&= 96 t^4
\end{align*}
\item Sei $0 \leq s < t$:
\begin{align*}
\E[X_t | \F_s] &= \E[W_t^4 | \F_s] - 3t^2 \\
&= \E[(W_t - W_s + W_s)^4 | \F_s] - 3t^2 \\
&= \E[(W_t - W_s)^4 + 4(W_t - W_s)^3W_s + 6(W_t - W_s)^2W_s^2 + \hdots \\
&\hspace{40pt} \hdots + 4(W_t - W_s)W_s^3 + W_s^4 | \F_s] - 3t^2 \\
&= \E[(W_t - W_s)^4] + 4\E[(W_t - W_s)^3]W_s + 6\E[(W_t - W_s)^2]W_s^2 + \hdots \\
&\hspace{40pt} \hdots + 4\E[(W_t - W_s)]W_s^3 + W_s^4 - 3t^2 \\
&= 3(t-s)^2 + 0 + 6(t-s)W_s^2 + 0 + W_s^4 - 3t^2 \\
&= 3(t-s)^2 + 6(t-s)W_s^2 + W_s^4 - 3t^2
\end{align*}
\item Nichteinmal ein Smartingal, da:
\begin{align*}
\E[X_t | \F_s] - X_s &= 3(t-s)^2 + 6(t-s)W_s^2 + W_s^4 - 3t^2 - W_s^4 + 3s^2 \\
&= 3(t-s)^2 + 6(t-s)W_s^2 - 3t^2 + 3s^2 \\
&= 6(t-s)W_s^2 + 3(t^2 - 2ts + s^2 - t^2 + s^2) \\
&= 6(t-s)W_s^2 + 3(-2ts +2 s^2) \\
&= 6( (t-s)W_s^2 - ts + s^2) \\
&= 6( (t-s)W_s^2 - s(t - s)) \\
&= 6(t-s)(W_s^2 - s)
\end{align*}
$W_s^2 - s$ ist ja ein Martingal laut Lévy, hat also beide Vorzeichen mit positiver Wahrscheinlichkeit.

\end{enumerate}

%9
\item Sei $\beta = e^{-\lambda}$
\begin{enumerate}
\item
\begin{align*}
\E[w_n] &= \E[X_n] - a \E[X_{n-1}] = 0 - 0 = 0 \\
\C[w_n, w_{n+h}] &= \C[X_n, X_{n+h}] - a \C[X_{n-1}, X_{n+h}] \hdots \\
&\hspace{40pt} \hdots- a \C[X_n, X_{n+h-1}] + a^2 \C[X_{n-1}, X_{n+h-1}] \\
&= (1+a^2)\beta^{|h|} - a(\beta^{|h+1|} + \beta^{|h-1|})
\end{align*}
\item Laut Wikipedia heißt ein Prozess White-Noise Prozess, i.e. $r_n \sim WN(\sigma^2)$ genau dann, wenn $\E[r_n] = 0$ und $\E[r_{n+h}r_n] = \sigma^2 \mathbbm 1_{\{0\}}(h)$ mit $\sigma^2 \neq 0$.

Vom Erwartungswert ist bereits gezeigt, das er $0$ ist. Desweitern, genau wegen der Zentriertheit:
\begin{align*}
\E[w_{n+h}w_n] &= \C[w_{n+h}, w_n] \stackrel{!}{=} \begin{cases}
\sigma^2 &h=0 \\ 0 & h \neq 0
\end{cases}
\end{align*}
Fall 1: $h \neq 0$, o.B.d.A. $h > 0$:
\begin{align*}
\E[w_{n+h}w_n] = (1+a^2)\beta^{h} - a(\beta^{h+1} + \beta^{h-1}) \stackrel{!}{=} 0 &\Leftrightarrow \\
(1+a^2) - a(\beta + \beta^{-1}) \stackrel{!}{=} 0 &\Leftrightarrow \\
a_{1,2} = \frac{\beta + \beta^{-1}}{2} \pm \sqrt{\frac{\beta^2 + 2 + \beta^{-2} - 4}{4}} \\
= \frac{\beta + \beta^{-1}}{2} \pm \frac{\beta - \beta^{-1}}{2} &\Leftrightarrow \\
a_1 = \beta, a_2 = \beta^{-1}
\end{align*}
Wir wählen $a = a_1 = \beta$ da $0 < \beta < 1$. Dann gilt $\C[w_{n+h}, w_n] = 0$.

Fall 2: $h = 0$:
\begin{align*}
\C[w_{n+0}, w_n] &= (1+a^2) a^0 - a(a^{|1|} + a^{|-1|}) \\
&= (1+a^2) - 2a^2 \\
&= 1 - a^2 =: \sigma ^2 \neq 0
\end{align*}
\qed

\end{enumerate}

%10
\item Ja, mit $c > 0$ beliebig und $a = \frac{1}{\sqrt c}$ ist $X_t = \frac{1}{\sqrt c}B_{ct}$ auch eine Brownsche Bewegung. Beweis mittels Lévy’s Martingalcharakterisierung: (Mit $t^* := ct, s^* := cs$)
\begin{itemize}
\item $X(0) = \frac{1}{\sqrt c} B(c\cdot 0) = 0$
\item X hat f.s stetige Pfade $\Leftrightarrow$ B hat f.s stetige Pfade
\item Martingalbedingung:
\begin{align*}
(X_t, t\geq0) \text{ ist ein Martingal} &\Leftrightarrow \E[X_t | \F_X(s)] = X_s &\forall t, s: s < t\\
&\Leftrightarrow \E[\frac{1}{\sqrt c}B_{ct} | \F_B(cs)] = \frac{1}{\sqrt c}B_{cs} &\forall t, s: s < t\\
&\Leftrightarrow \E[B_{t^*} | \F_B(s^*)] = B_{s^*} &\forall t^*, s^*: s^* < t^*\\
&\Leftrightarrow (B_{t^*}, t^*\geq0) \text{ ist ein Martingal}
\end{align*}
\item $(X_t^2 - t, t\geq0)$ ist ein Martingal $\Leftrightarrow (B_{t^*}^2 - t^*, t^*\geq 0)$ ist ein Martingal, da:
\begin{align*}
X_t^2 - t &= \frac{1}{c}B_{ct}^2 - \frac{ct}{c} \\
&= \frac{1}{c} (B_{t^*}^2 - t^*)
\end{align*} \qed

\end{itemize}




\end{enumerate}












\end{document}