\documentclass[a4paper,11pt,notitlepage,fullpage]{article}
%\documentclass{report}

\usepackage{fullpage}
\usepackage[utf8]{inputenc}
%\usepackage[ngerman]{babel}
%\usepackage[english]{babel}
\usepackage{amsmath}
\usepackage{amssymb}
\usepackage{latexsym}
\usepackage{mathtools}
\usepackage{listings}
\usepackage{bbm}
%\usepackage{algorithm}
%\usepackage{algpseudocode}
\usepackage{graphicx}
\usepackage{booktabs}
\usepackage{hhline}
\usepackage{amsthm}
\usepackage{cite}
\usepackage{wrapfig}
\usepackage{hyperref}
\usepackage{titling}
\usepackage{color}

\setlength{\droptitle}{-60pt}

\newcommand{\R}{\mathbb R}
\newcommand{\p}{\mathbb P}
\newcommand{\pp}[1]{\mathbb P\left[#1\right]}
\newcommand{\E}[1]{\mathbb E\left[#1\right]}
\newcommand{\V}{\mathbb V}
\newcommand{\Vv}[1]{\mathbb V\left[#1\right]}
\newcommand{\Cov}[1]{\mathbb Cov\left[#1\right]}
\newcommand{\F}{\mathcal{F}}
\newcommand{\ind}{\mathbbm{1}}
\newcommand{\indd}[1]{\mathbbm{1}_{#1}}
\newcommand{\norm}[2]{\left|\left|{#1}\right|\right|_{#2}}
\DeclareMathOperator*{\limm}{l.\hspace{-0.18em}i.\hspace{-0.19em}m}
\DeclareMathOperator*{\spann}{span}

\begin{document}
\author{Florian Bogner \& Alexander Palmrich}
\title{Stochastische Prozesse - Übung 12}
\maketitle

\begin{enumerate}
\setcounter{enumi}{50}

%%für ein Bild das copy-pasten und reinkommentieren
%\begin{figure}[h!]
%\centering
%\includegraphics[width=0.9\textwidth]{gfx/bildname.png}
%\label{fig1}
%\caption{TODO Beschreibung des Bildes}
%\end{figure}

%01
\item Zettel

%02
\item Zettel

%03
\item Zettel

%04
\item Wir betrachten das AR(2) Polynom (mit einer Variablenumbenennung) $$q(z) = 1 - az - bz^2$$ welches man durch $-b$ dividieren und umordnen kann. $$q(z) = z^2 + \frac{a}{b}z - \frac{1}{b}$$ Für die Stabilitätsbedingung sind wir interessiert an den Nullstellen. Die kleine Lösungsformel liefert uns eben diese. 
\begin{align*}
z_{1,2} &= -\frac{a}{2b} \pm \sqrt{\frac{a^2}{4b^2} + \frac{1}{b}} \\
&= -\frac{a}{2b} \pm \sqrt{\frac{a^2 + 4b}{4b^2}} \\
&= \frac{-a \pm \sqrt{a^2 + 4b}}{2b}
\end{align*}
Je nach Determinante $a^2+4b$ können nun die Nullstellen beide reell oder ein komplex konjungiertes Paar sein.
\begin{itemize}
\item Fall 1: $a^2 + 4b \geq 0$:
\begin{itemize}
\item[``$\Rightarrow$''] Das Polynom muss auf dem abgeschlossen Einheitsintervall positiv sein, also insbesondere auch an den Endpunkten 1 und -1. Daraus folgt jeweils die zweite und dritte Ungleichung:
\begin{align*}
q(1) = 1 - a - b > 0 &\Leftrightarrow b < 1 - a \\
q(-1) = 1 + a - b > 0 &\Leftrightarrow b < 1 + a \\
\end{align*}
Man kann das Polynom auch in Linearfaktoren faktorisieren:
$$q(z) = (z-z_1)(z-z_2) = z^2 - (z_1+z_2)z + z_1z_2 = z^2 + \frac{a}{b}z - \frac{1}{b}$$
ergo
$$-\frac{1}{b} = z_1 z_2$$
Da beide Nullstellen betragsweise größer 1 sein müssen, muss auch ihr Produkt betragsweise größer 1 sein:
$$\left|-\frac{1}{b}\right| > 1 \Rightarrow |b| < 1 \Rightarrow b > -1$$
Damit ist auch die erste Ungeichung gezeigt.

\item[``$\Leftarrow$'']
Beweis der Kontraposition: Angenommen mindestens eine Nullstelle liegt im Intervall $[-1, 1]$. Wir unterscheiden danach wie die Nullstellen verteilt sind.
\begin{itemize}
\item Fall 1.1: Es liegt genau eine (einfache) Nullstelle in $(0, 1]$. Da $q(0) = 1 > 0$ und das Vorzeichen genau einmal gewechselt wird, muss aber $$q(1) = 1 - a - b \leq 0 \Rightarrow b \geq 1 - a$$ Es ist also die zweite Gleichung verletzt.

\item Fall 1.2: Es liegt genau eine (einfache) Nullstelle in $[-1, 0)$. Selbes Spiel mit $q(-1)$, dritte Gleichung verletzt.

\item Fall 1.3: Es liegen beide Nullstellen (oder eine doppelte) in $(0, 1]$ oder beide in $[-1, 0)$. Deren Produkt ist liegt also definitiv in $(0, 1]$.
$$0 < z_1 z_2 = -\frac{1}{b} \leq 1 \Rightarrow b < 0 \wedge b \leq -1$$
In diesem Fall ist also die dritte Gleichung verletzt.

\end{itemize}

\end{itemize}

\item Fall 2: $a^2 + 4b < 0$: \\
Die Nullstellen sind also echt komplexe Zahlen und konjungiert zueinander. Sie haben insbesondere den gleichen Betrag und
$$|z_1|^2 = z_1 \bar z_1 = z_1 z_2 = -\frac{1}{b}$$
Man beachte das $a^2 + 4b < 0$ impliziert, dass $b < 0$. Damit ergibt sich:
$$|z_1|^2 > 1 \Leftrightarrow b > -1$$
Die Stabilitätsbedingung ist in diesem Fall also äquivalent zur ersten Gleichung. Die beiden anderen Gleichungen sind in diesem Fall allgemeingültig. Beweis durch Hinschauen.
%\begin{align*}
%a^2 + 4b &< 0 \\
%b < -\frac{a^2}{4} \leq 1-a
%\end{align*}
\end{itemize}
\qed

\end{enumerate}



\end{document}





















