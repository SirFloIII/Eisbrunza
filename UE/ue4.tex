\documentclass[a4paper,11pt,notitlepage,fullpage]{article}
%\documentclass{report}

\usepackage{fullpage}
\usepackage[utf8]{inputenc}
%\usepackage[ngerman]{babel}
%\usepackage[english]{babel}
\usepackage{amsmath}
\usepackage{amssymb}
\usepackage{latexsym}
\usepackage{mathtools}
\usepackage{listings}
\usepackage{bbm}
%\usepackage{algorithm}
%\usepackage{algpseudocode}
\usepackage{graphicx}
\usepackage{booktabs}
\usepackage{hhline}
\usepackage{amsthm}
\usepackage{cite}
\usepackage{wrapfig}
\usepackage{hyperref}
\usepackage{titling}
\usepackage{color}

\setlength{\droptitle}{-60pt}

\newcommand{\R}{\mathbb R}
\newcommand{\E}{\mathbb E}
\newcommand{\Ee}[1]{\mathbb E\left[#1\right]}
\newcommand{\V}{\mathbb V}
\newcommand{\Vv}[1]{\mathbb V\left[#1\right]}
\newcommand{\Cov}[1]{\mathbb Cov\left[#1\right]}
\newcommand{\ind}{\mathbbm{1}}
\newcommand{\indd}[1]{\mathbbm{1}_{#1}}

\begin{document}
\author{Florian Bogner \& Alexander Palmrich}
\title{Stochastische Prozesse - Übung 4}
\maketitle

\begin{enumerate}
\setcounter{enumi}{15}

%16
\item Laut VO 5, Folie 13 gilt $\E(\int_0^T Y(t)^2 dt) < \infty \Rightarrow Y \in M_T^2$. Wir wollen also die Vorraussetzung zeigen. Dabei verwenden wir den üblichen Fubini-Trick. Wir vertauschen erst unbegründet, sehen es kommt nicht unendlich heraus und wissen dadurch, dass das Vertauschen gerechtfertigt war.
\begin{align*}
\Ee{\int_0^T Y(t)^2 dt} &= \Ee{\int_0^T (W(t)-1)^4 dt} \\
&= \Ee{\int_0^T W(t)^4 - 4W(t)^3 + 6W(t)^2 - 4W(t) + 1 dt} \\
&= \int_0^T \Ee{W(t)^4} - 4\Ee{W(t)^3} + 6\Ee{W(t)^2} - 4\Ee{W(t)} + 1 dt &\text{Fubini-Trick} \\
&= \int_0^T 3t^2 - 0 + 6t - 0 + 1 dt \\
&= T^3 + 3T^2 + T < \infty &\text{für $T < \infty$}
\end{align*}
Desweiteren:
\begin{align*}
\Vv{Z(1)} &= \Ee{Z(1)^2} &\text{Folie 10, (25)} \\
&= \Ee{\left(\int_0^1 Y(s) dW(s)  \right)^2} \\
&= \Ee{\int_0^1 Y(s)^2  ds} &\text{Itô-Isometrie} \\
&= 1^3 + 3\cdot1^2 + 1  &\text{Siehe oben} \\
&= 5 &\square
\end{align*}


%17
\item Mittels Praxistipps auf VO5, Folie 10 berechnen wir ein paar Hilfsterme:
\begin{align*}
\Vv{X(t)} &= \Ee{X(t)^2} \\
&= \Ee{\int_0^t \sin(s)^2 ds} &\text{Itô-Isometrie} \\
&= \frac{t-\sin(t)\cos(t)}{2} &\text{deterministischer Erw.} \\
\Vv{Y(t)} &= \frac{t+\sin(t)\cos(t)}{2} &\text{analog}\\
\end{align*}
\begin{align*}
\Cov{X(t), Y(t)} &= \Ee{X(t) Y(t)}\\
&= \Ee{I_t (\sin) \cdot I_t (\cos)}\\
&= \Ee{\int_0^t \sin \cdot \cos ds}&\text{Itô-Isometrie} \\
&= \frac{\sin^2t}{2} &\text{deterministischer Erw.} \\
\end{align*}

Damit folgt nun
\begin{align*}
\rho(X(t), Y(t)) &= \frac{\Cov{X(t), Y(t)}}{\sqrt{\Vv{X(t)}\Vv{Y(t)}}} \\
&= \frac{\frac{\sin^2t}{2}}{\sqrt{ \frac{t^2 - \sin(t)^2\cos(t)^2}{4} }} \\
&= \frac{\sin^2t}{\sqrt{ t^2 - \sin(t)^2\cos(t)^2}}  &\square\\
\end{align*}

%18
\item Wir überprüfen die Voraussetzungen aus VO 5, Folie 30.
\begin{align*}
F(t, x) &= S_0 \exp\left(\left(\mu - \frac{\sigma^2}{2}\right)t + \sigma x\right) \\
F_t(t, x) &= S_0 \exp\left(\left(\mu - \frac{\sigma^2}{2}\right)t + \sigma x\right) \cdot \left(\mu - \frac{\sigma^2}{2}\right) = \left(\mu - \frac{\sigma^2}{2}\right) F(t, x)\\
F_x(t, x) &= S_0 \sigma \exp\left(\left(\mu - \frac{\sigma^2}{2}\right)t + \sigma x\right) = \sigma F(t, x) \\
F_{xx}(t, x) &= S_0 \sigma^2 \exp\left(\left(\mu - \frac{\sigma^2}{2}\right)t + \sigma x\right) = \sigma^2 F(t, x)
\end{align*}
Es muss gelten, dass $(F_x(t, W(t)), t \geq 0) \in M_T^2$ für alle $T > 0$. Beweis wie in 16):
\begin{align*}
\Ee{\int_0^T F_x(t, W(t))^2 dt} &= \Ee{\int_0^T \sigma^2 S_0^2 \exp(2(\mu - \frac{\sigma^2}{2}) t) \cdot \exp(2\sigma W(t)) dt} \\
&\leq  \sigma^2 S_0^2 \|\exp(2(\mu - \frac{\sigma^2}{2}) t)\|_{L_\infty[0, T]} \Ee{\int _0^T \exp(2\sigma W(t)) dt} \\
&= c \cdot \Ee{\int _0^T \exp(2\sigma W(t)) dt} &c < \infty
\end{align*}
Aus AmStat wissen wir: Wenn $X\sim N(\mu, \sigma^2)$, dann $e^X \sim LogN(\mu, \sigma^2)$ und $\Ee{e^X} = exp(\mu + \frac{\sigma^2}{2})$. In unserem Fall: $2\sigma W(t) \sim N(0, 4\sigma^2 t)$ und damit $\Ee{e^{2\sigma W(t)}} =  exp(2\sigma^2 t)$. Wieder mit spekulativem Fubini-Trick folgt also:
\begin{align*}
c \cdot \Ee{\int _0^T \exp(2\sigma W(t)) dt} &= c \int _0^T \Ee{\exp(2\sigma W(t))} dt \\
&= c \int _0^T exp(2\sigma^2 t) dt \\
&= c \frac{e^{2\sigma^2 T}-1}{2\sigma^2} < \infty
\end{align*}
Also gilt nun:
\begin{align*}
\xi(T) &= F(T, W(T)) \\
&= F(0, W(0)) + \int_0^T F_t(t, W(t)) + \frac{1}{2} F_{xx}(t, W(t)) dt + \int_0^T F_x(t, W(t) dW(t) \\
&= S_0 + \int_0^T \mu F(t, W(t)) dt + \int_0^T \sigma F(t, W(t)) dW(t) \text{ ist Itô-Prozess.}&\square
\end{align*}

%19
\item Ganz analog:
\begin{align*}
F(t, x) &= e^{tx} \\
F_t(t, x) &= x e^{tx} \\
F_x(t, x) &= t e^{tx} \\
F_{xx}(t, x) &= t^2 e^{tx}
\end{align*}
Nochmals $(F_x(t, W(t)), t \geq 0) \in M_T^2$ für alle $T > 0$, 
\begin{align*}
\Ee{ \int_0^T F_x(t, W(t))^2 dt} &= \Ee{ \int_0^T t^2 e^{2tW(t)} dt} \\
&= \int_0^T \Ee{ t^2 e^{2tW(t)}} dt &\text{Fubini-Trick} \\
&= \int_0^T  t^2 \Ee{ e^{2tW(t)}} dt \\
&= \int_0^T  t^2 e^{2t^3} dt &\text{Erw. der Lognormalvert.}\\
&= \frac{e^{2T^3} - 1}{6} < \infty
\end{align*}
Dementsprechend ist $\xi$ ein Itô-Prozess.
\begin{align*}
\xi(T) &= F(T, W(T)) \\
&= F(0, W(0)) + \int_0^T F_t(t, W(t)) + \frac{1}{2} F_{xx}(t, W(t)) dt + \int_0^T F_x(t, W(t) dW(t) \\
&= 1 + \int_0^T (W(t) + \frac{t^2}{2}) e^{tW(t)} dt + \int_0^T t e^{tW(t)} dW(t)&\square
\end{align*}

%20
\item TODO Bsp 20


\end{enumerate}












\end{document}
