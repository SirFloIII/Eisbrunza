\documentclass[a4paper,11pt,notitlepage,fullpage]{article}
%\documentclass{report}

\usepackage{fullpage}
\usepackage[utf8]{inputenc}
%\usepackage[ngerman]{babel}
%\usepackage[english]{babel}
\usepackage{amsmath}
\usepackage{amssymb}
\usepackage{latexsym}
\usepackage{mathtools}
\usepackage{listings}
\usepackage{bbm}
%\usepackage{algorithm}
%\usepackage{algpseudocode}
\usepackage{graphicx}
\usepackage{booktabs}
\usepackage{hhline}
\usepackage{amsthm}
\usepackage{cite}
\usepackage{wrapfig}
\usepackage{hyperref}
\usepackage{titling}
\usepackage{color}

\setlength{\droptitle}{-60pt}

\newcommand{\R}{\mathbb R}
\newcommand{\E}{\mathbb E}
\newcommand{\V}{\mathbb V}
\newcommand{\ind}{\mathbbm{1}}

\begin{document}
\author{Florian Bogner \& Alexander Palmrich}
\title{Stochastische Prozesse - Übung 4}
\maketitle

\begin{enumerate}
\setcounter{enumi}{15}

%16
\item Laut VO 5, Folie 13 gilt $\E(\int_0^T Y(t)^2 dt) < \infty \Rightarrow Y \in M_T^2$. Wir wollen also die Vorraussetzung zeigen:
\begin{align*}
\E\left(\int_0^T Y(t)^2 dt\right) &= \E\left(\int_0^T (W(t)-1)^4 dt\right) \\
&= \E\left(\int_0^T W(t)^4 + 4W(t)^3 + 6W(t)^2 + 4W(t) + 1 dt\right) \\
&= \int_0^T \E(W(t)^4) + 4\E(W(t)^3) + 6\E(W(t)^2) + 4\E(W(t)) + 1 dt &\text{Fubini\footnote{Der übliche Trick, wir vertauschen erst unbegründet, sehen es kommt nicht unendlich heraus und wissen dadurch, dass das Vertauschen gerechtfertigt war.}} \\
&= \int_0^T 3t^2 + 0 + 6t + 0 + 1 dt \\
&= T^3 + 3T^2 + T < \infty &\text{für $T < \infty$}
\end{align*}
Desweiteren:
\begin{align*}
\V(Z(1)) &= \E(Z(1)^2) &\text{Folie 10, (25)} \\
&= \E\left(   \left(\int_0^1 Y(s) dW(s)  \right)^2  \right) \\
&= \E\left(  \int_0^1 Y(s)^2  ds  \right) &\text{Itô-Isometrie} \\
&= 1^3 + 3\cdot1^2 + 1  &\text{Siehe oben} \\
&= 5
\end{align*}












\end{enumerate}












\end{document}