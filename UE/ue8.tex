\documentclass[a4paper,11pt,notitlepage,fullpage]{article}
%\documentclass{report}

\usepackage{fullpage}
\usepackage[utf8]{inputenc}
%\usepackage[ngerman]{babel}
\usepackage[english]{babel}
\usepackage{amsmath}
\usepackage{amssymb}
\usepackage{latexsym}
\usepackage{mathtools}
\usepackage{listings}
\usepackage{algorithm}
\usepackage{algpseudocode}
\usepackage{graphicx}
\usepackage{booktabs}
\usepackage{hhline}
\usepackage{amsthm}
\usepackage{cite}
\usepackage{wrapfig}
\usepackage{hyperref}
\usepackage{titling}
\usepackage{color}

\setlength{\droptitle}{-60pt}


\begin{document}
\author{Florian Bogner}
\title{Stochastische Prozesse - Übung 8}
\maketitle

\begin{enumerate}
\setcounter{enumi}{34}

%35
\item 

%36
\item

%37
\item Laut Formeln aus der VO haben wir:
\begin{align*}
\gamma(0) &= \sigma^2 \cdot (b_0^2 + b_1^2) \\
\gamma(1) &= \sigma^2 \cdot b_0 \cdot b_1 \\
\rho(1) &= \frac{b_0 \cdot b_1}{b_0^2 + b_1^2}
\end{align*}
Mit dem Binomialsatz ergibt sich:
\begin{align*}
0 &\leq (b_0 \pm b_1)^2 \\
&= b_0^2 \pm 2b_0b_1 + b_1^2 \\
\Leftrightarrow \mp 2b_0b_1 &\leq b_0^2 + b_1^2 \\
\Leftrightarrow \mp \frac{b_0 \cdot b_1}{b_0^2 + b_1^2} &\leq \frac{1}{2} \\
\Rightarrow |\rho(1)| &\leq \frac{1}{2}
\end{align*}




%38
\item

\end{enumerate}


















\end{document}