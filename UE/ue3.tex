\documentclass[a4paper,11pt,notitlepage,fullpage]{article}
%\documentclass{report}

\usepackage{fullpage}
\usepackage[utf8]{inputenc}
%\usepackage[ngerman]{babel}
\usepackage[english]{babel}
\usepackage{amsmath}
\usepackage{amssymb}
\usepackage{latexsym}
\usepackage{mathtools}
\usepackage{listings}
\usepackage{algorithm}
\usepackage{algpseudocode}
\usepackage{graphicx}
\usepackage{booktabs}
\usepackage{hhline}
\usepackage{amsthm}
\usepackage{cite}
\usepackage{wrapfig}
\usepackage{hyperref}
\usepackage{titling}
\usepackage{color}

\setlength{\droptitle}{-60pt}

\newcommand{\R}{\mathbb R}

\begin{document}
\author{Florian Bogner \& Alexander Palmrich}
\title{Stochastische Prozesse - Übung x}
\maketitle

\begin{enumerate}
\setcounter{enumi}{10}

%11
\item bla

%12
\item bla

%13
\item bla

%14
\item bla

%15
\item Angenommen $W$ ist eine Brownsche Bewegung und $f: \R_+ \to \R$ ist eine beschränkte stetige Funktion und $0 \leq t_0 < t_1 < \cdots < t_n$. Welche Verteilung hat die Zufallsvariable
\begin{align*}
U = \sum_{k=1}^n f(t_{k-1})\Delta W(t_k)
\end{align*}
Die vorkommenden Inkremente von $W$ sind bekanntlich unabhängig normalverteilt, i.e. $\Delta W(t_k) \sim N(0, t_k - t_{k-1})$. Demnach ist $U$ eine Linearkombination aus Normalverteilungen und nach dem Reproduktionssatz wieder eine Normalverteilung. Mit
\begin{align*}
A &:= (f(t_0), f(t_1), \cdots, f(t_{n-1}) \in \R^{1\times n} \\
\mu &:= (0, 0, \cdots, 0)^T \in R^n \\
\sigma &:= diag(t_1 - t_0, t_2 - t_1, \cdots, t_n - t_{n-1}) \in \R^{n\times n}
\end{align*}
folgt:
\begin{align*}
U &\sim N\left(A\mu, A^T \sigma A\right) \\
&\sim N\left(0, \sum_{k=1}^n f(t_{k-1})^2(t_k - t_{k-1})\right)
\end{align*}
Erwartungswert und Varianz stehen damit schon da.


\end{enumerate}












\end{document}